\documentclass[9pt]{extarticle} % defines the kind of document you want to produce

% Include different packages:
\usepackage[margin=0.75cm]{geometry} % set margins
\usepackage{multicol}
\usepackage[T1]{fontenc} % for correct hyphenation and T1 encoding
\usepackage{lmodern}
\usepackage[english]{babel}
\usepackage{amsmath}
\usepackage{amssymb}
\usepackage{mathtools}
\usepackage{graphicx}           	% include graphics
\usepackage{caption}
\usepackage{subcaption}
\usepackage{hyperref}
\usepackage{mdframed}
\usepackage{tabularx}
\usepackage{wasysym}

\usepackage{listings}
\usepackage{color, xcolor}
\usepackage{blindtext}

%\DeclareMathSizes{10}{17}{12}{26}

%# Colors ------------------------------------------------------------------------
\definecolor{mGreen}{rgb}{0,0.6,0}
\definecolor{mGray}{rgb}{0.5,0.5,0.5}
\definecolor{mPurple}{rgb}{0.58,0,0.82}
\definecolor{backgroundColour}{rgb}{0.95,0.95,0.92}
\definecolor{DENcol}{RGB}{35,171,196} % Department Colour
\definecolor{DENcol10}{RGB}{245, 245, 245}

%# Listings -----------------------------------------------------------------------
\lstMakeShortInline{|}

\lstloadlanguages{Bash,VHDL,Matlab,[ANSI]C,Java,[LaTeX]TeX,Python}

\lstset{
    frame=top,frame=bottom,
    basicstyle=\fontsize{11}{11}\normalfont\sffamily,
    commentstyle=\color{mGreen},
    keywordstyle=\color{magenta},
    numberstyle=\tiny\color{DENcol},
    stringstyle=\color{mPurple},            % the size of the fonts that are used for the code
	stepnumber=1,                           % the step between two line-numbers. If it is 1 each line will be numbered
	numbersep=6pt,                          % how far the line-numbers are from the code
	tabsize=2,                              % tab size in blank spaces
	extendedchars=true,                     %
	breaklines=true,                        % sets automatic line breaking
	captionpos=t,                           % sets the caption-position to top
	mathescape=true,
	stringstyle=\color{black}\ttfamily,     % Farbe der String
	showspaces=false,                       % Leerzeichen anzeigen ?
	showtabs=false,                         % Tabs anzeigen ?
	xleftmargin=15pt,
	framexleftmargin=14pt,
	framexrightmargin=9pt,
	framexbottommargin=9pt,
	framextopmargin=5pt,
	showstringspaces=false,      % Leerzeichen in Strings anzeigen ?
	numbers = left,
	linewidth = 150mm,
	language=Python
}

%# MDFramed Box style -------------------------------------------------------------
% Red box
\mdfdefinestyle{redbox}{%
		linecolor=red,
		linewidth=1pt,
}

% Equation box
\mdfdefinestyle{eqbox}{%
		linecolor=black,
		linewidth=1pt,
		innerrightmargin=2pt,
		innerleftmargin=2pt
		innertopmargin=-2pt,
		innerbottommargin=2pt,
		backgroundcolor=DENcol10,
}

% Blue box
\mdfdefinestyle{bluebox}{%
		linecolor=DENcol,
		linewidth=1pt,
		backgroundcolor=DENcol10
}

%# Font Style ---------------------------------------------------------------------
% Reduce space between section and subsection lines
\usepackage{titlesec}
\titleformat{\section}
	{\normalfont\large\bfseries}{\thesection}{1em}{}
\titlespacing*{\section}{2pt}{*0}{2pt}
\titleformat{\subsection}
	{\normalfont\normalsize\bfseries}{\thesubsection}{.2em}{}
\titlespacing*{\subsection}{0pt}{0pt}{0pt}
\titleformat{\subsubsection}
	{\normalfont\normalsize\bfseries}{\thesubsubsection}{.2em}{}
\titlespacing*{\subsubsection}{0pt}{1pt}{0pt}

%\RedeclareSectionCommand[
%	beforeskip=.75\baselineskip,
%	afterskip=.5\baselineskip]{section}
%\RedeclareSectionCommand[
%	beforeskip=0\baselineskip,
%	afterskip=0\baselineskip]{subsection}

\makeatletter
\def\@maketitle{
	\begin{center}
		{\rmfamily\Large\mdseries\@title} - {\sffamily\large\@author} - {\rmfamily\large\mdseries\@date} 
	\end{center}\vspace*{1em}
}
\makeatother

%# Commands -----------------------------------------------------------------------
% Matrix in Mathmode
\newcommand{\mat}[1]{\mathbf{#1}}
