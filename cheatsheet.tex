\documentclass[9pt]{extarticle} % defines the kind of document you want to produce

% Include different packages:
\usepackage[margin=0.75cm]{geometry} % set margins
\usepackage{multicol}
\usepackage[T1]{fontenc} % for correct hyphenation and T1 encoding
\usepackage{lmodern}
\usepackage[english]{babel}
\usepackage{amsmath}
\usepackage{amssymb}
\usepackage{mathtools}
\usepackage{graphicx}           	% include graphics
\usepackage{caption}
\usepackage{subcaption}
\usepackage{hyperref}
\usepackage{mdframed}
\usepackage{tabularx}
\usepackage{wasysym}

\usepackage{listings}
\usepackage{color, xcolor}
\usepackage{blindtext}

%\DeclareMathSizes{10}{17}{12}{26}

%# Colors ------------------------------------------------------------------------
\definecolor{mGreen}{rgb}{0,0.6,0}
\definecolor{mGray}{rgb}{0.5,0.5,0.5}
\definecolor{mPurple}{rgb}{0.58,0,0.82}
\definecolor{backgroundColour}{rgb}{0.95,0.95,0.92}
\definecolor{DENcol}{RGB}{35,171,196} % Department Colour
\definecolor{DENcol10}{RGB}{245, 245, 245}

%# Listings -----------------------------------------------------------------------
\lstMakeShortInline{|}

\lstloadlanguages{Bash,VHDL,Matlab,[ANSI]C,Java,[LaTeX]TeX,Python}

\lstset{
    frame=top,frame=bottom,
    basicstyle=\fontsize{11}{11}\normalfont\sffamily,
    commentstyle=\color{mGreen},
    keywordstyle=\color{magenta},
    numberstyle=\tiny\color{DENcol},
    stringstyle=\color{mPurple},            % the size of the fonts that are used for the code
	stepnumber=1,                           % the step between two line-numbers. If it is 1 each line will be numbered
	numbersep=6pt,                          % how far the line-numbers are from the code
	tabsize=2,                              % tab size in blank spaces
	extendedchars=true,                     %
	breaklines=true,                        % sets automatic line breaking
	captionpos=t,                           % sets the caption-position to top
	mathescape=true,
	stringstyle=\color{black}\ttfamily,     % Farbe der String
	showspaces=false,                       % Leerzeichen anzeigen ?
	showtabs=false,                         % Tabs anzeigen ?
	xleftmargin=15pt,
	framexleftmargin=14pt,
	framexrightmargin=9pt,
	framexbottommargin=9pt,
	framextopmargin=5pt,
	showstringspaces=false,      % Leerzeichen in Strings anzeigen ?
	numbers = left,
	linewidth = 150mm,
	language=Python
}

%# MDFramed Box style -------------------------------------------------------------
% Red box
\mdfdefinestyle{redbox}{%
		linecolor=red,
		linewidth=1pt,
}

% Equation box
\mdfdefinestyle{eqbox}{%
		linecolor=black,
		linewidth=1pt,
		innerrightmargin=2pt,
		innerleftmargin=2pt
		innertopmargin=-2pt,
		innerbottommargin=2pt,
		backgroundcolor=DENcol10,
}

% Blue box
\mdfdefinestyle{bluebox}{%
		linecolor=DENcol,
		linewidth=1pt,
		backgroundcolor=DENcol10
}

%# Font Style ---------------------------------------------------------------------
% Reduce space between section and subsection lines
\usepackage{titlesec}
\titleformat{\section}
	{\normalfont\large\bfseries}{\thesection}{1em}{}
\titlespacing*{\section}{2pt}{*0}{2pt}
\titleformat{\subsection}
	{\normalfont\normalsize\bfseries}{\thesubsection}{.2em}{}
\titlespacing*{\subsection}{0pt}{0pt}{0pt}
\titleformat{\subsubsection}
	{\normalfont\normalsize\bfseries}{\thesubsubsection}{.2em}{}
\titlespacing*{\subsubsection}{0pt}{1pt}{0pt}

%\RedeclareSectionCommand[
%	beforeskip=.75\baselineskip,
%	afterskip=.5\baselineskip]{section}
%\RedeclareSectionCommand[
%	beforeskip=0\baselineskip,
%	afterskip=0\baselineskip]{subsection}

\makeatletter
\def\@maketitle{
	\begin{center}
		{\rmfamily\Large\mdseries\@title} - {\sffamily\large\@author} - {\rmfamily\large\mdseries\@date} 
	\end{center}\vspace*{1em}
}
\makeatother

%# Commands -----------------------------------------------------------------------
% Matrix in Mathmode
\newcommand{\mat}[1]{\mathbf{#1}}


%# Titlepage ---------------------------------------------------------------------
\title{Fundamentals of Foundation Models\\Cheat sheet}
\author{Lukas Schüttler}
\date{\today}
\pagestyle{empty}

%# Document ----------------------------------------------------------------------
\begin{document}
\maketitle
\thispagestyle{empty}
\begin{multicols}{2}
    \section{General}
    \begin{mdframed}[style=eqbox]
        \textbf{SGD Updates} - $\theta_{k+1} - \theta_{k} - \sum_{i \in \mathcal{B}} \nabla \operatorname{loss}(f_{\theta_k}, i)$\\
        {\tiny With $\mathcal{B}$ being a batch os size $B$}\\
        \textbf{ARM} - Auto regressive model
    \end{mdframed}
    \section{Transformers}
    \begin{mdframed}[style=eqbox]
        \textbf{Token Embeddings}\\
        $\mat{I} \in \mathbb{R}^{\mathrm{voc}\times N}$: Input Sequence - $\mat{M_{\mathrm{emb}}} \in \mathbb{R}^{\mathrm{dim\_emb} \times \mathrm{voc}}$: Embedding Matrix - $\mat{M_{\mathrm{pos}}} \in \mathbb{R}^{\mathrm{dim\_emb} \times \mathrm{N}}$: Positional Embedding - $\mat{E} \in \mathbb{R}^{\mathrm{dim\_emb} \times \mathrm{N}}$: Token Embeddings
        \begin{align*}
            \mat{E} = \mat{M}_\mathrm{emb} \times \mat{I} + \mat{M}_\mathrm{pos}
        \end{align*}
        \textbf{Self Attention}\\
        $\mat{W} = \left[ \mat{W}_q ~ \mat{W}_k ~ \mat{W}_v \right]^\top \in \mathbb{R}^{3\mathrm{dim\_emb} \times \mathrm{dim\_emb}}$ - $\mat{Q}$, $\mat{K}$, $\mat{V}$ $\in \mathbb{R}^{\mathrm{dim\_emb} \times N}$: Query, Key, Value - $\mat{A} \in \mathbb{R}^{N \times N}$: Attention Matrix
        \begin{align*}
            \left[ \mat{Q} ~ \mat{K} ~ \mat{V} \right]^\top &= \mat{W} \times \mat{E}\\
            \mat{A} &= \operatorname{softmax}\left( \frac{\mat{Q}^\top \mat{K}}{\sqrt{d_k}}\right)~\astrosun~ \mat{M}_\mathrm{mask}\\
            \mat{E}_\mathrm{att} &= \mat{V} \times \mat{A}
        \end{align*}
        {\tiny With $d_k$ being the dimension of the key vectors. E.g. $\mathrm{pos\_emb}$ for single head attention}\\
        \textbf{MLP}\\
        $\mat{M}_\mathrm{up} \in \mathbb{R}^{4\mathrm{dim\_emb} \times \mathrm{dim\_emb}}$: Up projection - $\mat{M}_\mathrm{down} \in \mathbb{R}^{\mathrm{dim\_emb} \times 4\mathrm{dim\_emb}}$: Down projection
        \begin{align*}
            E_\mathrm{mlp} = \sigma(\mat{M}_\mathrm{down} \times \sigma (\mat{M}_\mathrm{up} \times \mat{E}_\mathrm{att})))
        \end{align*}
        {\tiny With $\sigma$ being the activation applied elementwise}\\
        \textbf{Computation:} $\mathrm{FLOPS} \approx 6N \cdot D$ - With $D$ being the number of training tokens.
    \end{mdframed}

    \section{Tokenization}
    \begin{mdframed}[style=eqbox]
        \textbf{Pre-tokenization}: Split input text via regex to prevent greedy encoding of words in different commen variations: \textid{dog? dog!}. Additionally tokens lead with a space: ' \textit{and}'.\\
        \subsection{Tokenizers}
        \textbf{Byte Pair Encoding (BPE)} - Entropy encoding - Given a base vocabulary size of 256 with UTF-8 BPE encodes additonal pairs of bytes as subsequent numbers to 256 eg. 257, 258...\\
        Pairs are encoded based on their frequency of occurrence to achieve the maximum compression. The process is iteratively repeated on a vocabulary set until the maximum voc. size is reached (e.g. GPT-4 ~100k) or the algorithm runs out of pairs.\\
        \textbf{WordPiece} - Pointwise mutual information
    \end{mdframed}
    \begin{mdframed}[style=eqbox]
        \subsection{Byte Pair Encoding}
    \end{mdframed}

    \section{Training}
    \begin{mdframed}[style=eqbox]
        \subsection{Evaluation}
        \textbf{Perplexity}: $\exp\left( - \cfrac{1}{N} \sum_{i=1}^{N} \log P(t_i)\right)$\\
        {\tiny With $t_i$ being the $i$th token in the expected output sequence}\\[1em]
        \textbf{Benchmarks}\\
        \textbf{Paloma}: Perplexity over a diverse set to text.\\
        \textbf{HellaSwag}: QA Benchmark. Most likely answer by perplexity is chosen.\\
        \textbf{MMLU}: QA Benchmark. Answers are part of the Prompt. Model can answer A, B, C or D. Most likely token is chosen.
    \end{mdframed}

    \begin{mdframed}[style=eqbox]
        \subsection{Datasets}
        \textbf{Common Crawl}: Database of scraped websites.
        \textbf{WebText}: OpenAI internal dataset. Scraped links from Reddit which received at least 3 karma. Page de-duplication and light-cleaning. \textit{8M} Documents, \textit{40GB} of text.\\
        \textbf{OpenWebText}: Open replication of \textbf{WebText}\\
        \textbf{C4}: Colossal Clean Crawled Corpus. Filtered version of \textbf{Common Crawl}. Discard pages with fewer than 5 sentences and lines with fewer than 3 words. Filter for unwanted keywords. Remove lines with the word Javascript. Remove pages with "\textit{lorem ipsum}". Remove pages with "\textit{\{}". Deduplicate any three-sentence span occurring multiple times. Filter pages that are not in English.\\
        \textbf{The Stack}: Coding dataset.\\
        \textbf{PeS2o}: STEM papers.\\
        \textbf{DOLMA}: Combination of \textbf{Common Crawl}, \textbf{C4}, \textbf{The Stack}, \textbf{Reddit}, \textbf{PeS2o}, \textbf{Project Gutenberg}, \textbf{Wikipedia/Wikibooks}\\
        \textbf{The Pile}: \textit{800GB} Dataset of Diverse Text. Academic, Internet, Prose, Dialoque and Misc (GitHub, Math ...)\\[1em]
        \textbf{Dataset Cleaning Pipeline}: Language Filtering $\to$ Deduplication (by URL) $\to$ Quality Filters $\to$ Content Filters $\to$ Deduplication (by text overlap)
    \end{mdframed}

    \begin{mdframed}[style=eqbox]
        \subsection{Fine-Tuning}
        Possible with around $1000$ high quality prompts and responses or more.\\
        \textbf{RLHF} - Reinforcement Learning from Human Feedback
        \begin{align*}
            y_1,y_2 \propto \pi_{\mathrm{SFT}}(y \mid x) && y_w > y_\ell \mid x
        \end{align*}
        \vspace{-2em}
        \begin{align*}
            p^*(y_1 > y_2 \mid x) &= \frac{1}{1 + \exp(r^*(x \mid y_2) - r^*(x \mid y_1))}\\
            \mathcal{L}(r) &= - \mathbb{E} \left[\log\sigma (r(x \mid y_\ell) - r(x \mid y_w)) \right]
        \end{align*}
        {\tiny Where $r$ is the reward model and $\mathcal{L}$ is the loss of the reward model.}
        \begin{align*}
            \max_\pi \mathbb{E}\left[r(x,y) - \beta \operatorname{D}_\mathrm{KL} (\pi(y \mid x) \mid \pi_\mathrm{ref} (y \mid x)\right]
        \end{align*}
        {\tiny $\operatorname{D}_\mathrm{KL}$ to reduce the deviation from the base model (SFT).\\}
        \textbf{DPO} - Direct Preference Optimization
        \begin{align*}
            \mathcal{L}_\mathrm{DPO} &= - \mathbb{E} \left[ \log \sigma \left( \beta \log \frac{\pi_\theta (y_w \mid x)}{\pi_\mathrm{ref} (y_w \mid x)} - \beta \log \frac{\pi_\theta (y_\ell \mid x)}{\pi_\mathrm{ref} (y_\ell \mid x)} \right) \right]
        \end{align*}
    \end{mdframed}

    \section{Ensembles and MoE}
    \begin{mdframed}[style=eqbox]
        \subsection{Ensembles}
        \textbf{Linear interpolation} - $P(y \mid x) = \sum_m P_m(y \mid x) P(m \mid x)$\\
        \textbf{Log-linear interpolation} - $\operatorname{softmax}\left(\sum_m \log P(y \mid x) \lambda_m(x)\right)$\\
        With $P_m(y \mid x)$ being the output of the model $m$ and $P(m \mid x)$ the "reliability" of the model given the Input. $\lambda_m$ is an interpolation coefficients for the model $m$.\\
        \textbf{Parameter Averaging} - Calculate model weights by accumulating (average, weighted average ...) weights from multiple models
    \end{mdframed}
    \newpage

    \begin{mdframed}[style=eqbox]
        \subsection{MoE - Mixture of Experts}
        \textbf{Gaussian mixture model} - $p(y \mid x) = \sum_k p_{\theta_k}(y) p_k(x)$\\
        Where $p_{\theta_k}$ is a gaussian distribution parametrized by $\theta_k$ giving the propabillity of the output $y$. $p_k$ is the probability of that distribution given the input $x$.\\
        Allows the approximation of more complex distributions using only simple distributions (gaussian and logistic)\\[0.5em]
        \textbf{Routing}\\
        \textbf{Shazeer} -
        \textbf{Mixtral} -
        \textbf{Switch routing} -
    \end{mdframed}

    \section{Scaling Laws}
    \begin{mdframed}[style=eqbox]
        \begin{align*}
            L(D) \approx \frac{A}{D^{\alpha_D}} && L(N) \approx \frac{B}{N^{\alpha_N}}
        \end{align*}\vspace{-2em}
        \begin{align*}
            R(f_{N,D}) &= R(f^*) + (R(f_N) - R(f^*)) + (R(f_{N,D}) - R(f_N))\\
            R(f_{N,D}) &= E + L(N) + L(D)
        \end{align*}
        {\small
        Let $R(f^*)$ be the irreducible error, $R(f_N) - R(f^*)$ the approximation error of a N-parameter model and $R(f_{N,D}) - R(f_N)$ the statistical error\\
        Parameter Values depend on data and precise model architecture.
        }
        \subsection{Compute Optimality}
        $\arg\min L(N, D)$ for a constant compute budget $C(N, D) = H$ by choosing the best $N$ and $D$.\\
        \textbf{Training curve envelope} - Plot training curves ($x$: $\log(\mathrm{FLOPS})$, $y$: $L(N,D)$), Find the envelope (Minimal loss per FLOP), Plot envelope points twice ($x$: $\log(\mathrm{FLOPS})$, $y$: $\log(D)$, $\log(N)$), Fit a line to the points and extrapolate to the desired FLOPS to find the optimal $N$ and $D$.\\
        \textbf{IsoFLOP Curves} - Train various model sizes with $D$ such that the final FLOPs are constant, Repeat for different final FLOPs, Plot final loss (x: $\log(N)$, y: $L(N, D)$), Locate optimal model size for a given compute budget (loss valley), Plot optimal models ($x$: $\log(\mathrm{FLOPS})$, $y$: $\log(D)$, $\log(N)$) and extrapolate.\\
        \textbf{Parametric fit} - Fit parametric Risk function $R(f_{N,D})$ to the training results (Training like IsoFLOP), Plot contours ($x$: $\log(\mathrm{FLOPS})$, $y$: $\log(N)$), Fit line such that it goes through each iso-loss contour at the point with the fewest FLOPs. Extrapolate to desired FLOPs.\\
        Alternative - Plot isoFLOP slice (x: $\log(N)$, y: $L(N, D)$), plot parametric risk for desired compute budget, Locate the minimum.\\[0.5em]
        \textbf{Key finding:} For compute optimal training $D$ should scale proportionally with $N$ - Compute Optimality doesn't consider inference cost
    \end{mdframed}


    \section{Scalable Computing}
    \begin{mdframed}[style=eqbox]
        \subsection{CUDA}
        \textbf{CUDA Thread} - Smallest compute entity\\
        \textbf{CUDA Block} - Group of threads (up to $1024$)\\
        \textbf{Streaming Multiprocessor} - Executes one \textbf{CUDA Block}
    \end{mdframed}
    \begin{mdframed}[style=eqbox]
        \subsection{Multicore Processing}
        GPUs are optimized for performing the same operation on different data simultaneously (\textbf{SIMD} - single-instruction multiple-data)\\
        \textbf{Amdahl's law} - Let the non-parallizable part of a program take a fraction $s$ of the time, then $m$ workers can result in a speedup of:\\\vspace{-1.5em}
        \begin{align*}
            \frac{1}{s + \frac{1 - s}{m}}
        \end{align*}
        \textbf{Floating Point Numbers}\\
        \begin{tabular}{r|ccc}
            & Sign & Exponential & Mantissa \\
            \hline
            \textbf{FP32} & 1 Bit & 8 Bits & 23 Bits\\
            \textbf{FP16} & 1 Bit & 5 Bits & 10 Bits\\
            \textbf{BF16} & 1 Bit & 8 Bits & 7 Bits
        \end{tabular}\\[0.5em]
        \textbf{BF16} - \textbf{FP32} range with \textbf{FP16} precision\\
        \textbf{Error:} $\mid x - \widetilde{x} \mid \leq \epsilon / 2 \mid x \mid && \mathrm{with~} \epsilon \neq 0$
    \end{mdframed}

    \begin{mdframed}[style=eqbox]
        \subsection{Distributed Computing}
        \textbf{Communication Patterns}\\
        \textbf{Push} - $A$ sends data to $B$\\
        \textbf{Pull} - $B$ requests data from $A$\\
        \textbf{Reduce} - Data from $C_1, \dots C_n$ is aggregated on $A$ (sum, average, min, max \dots)\\
        \textbf{All-Reduce} - Reduce but result is distributed back to $C_1, \dots C_n$\\
        \textbf{Scatter-Reduce} - \dots\\
        \textbf{Wait} - $A$ waits fo a signal from $B$\\
        \textbf{Barrier} - $C_1, \dots C_n$ wait until all $C_i$ reach a point of execution and then continue\\[0.5em]
        \textbf{Training Methods}\\
        \textbf{Synchronous} - Each machine computes a minibatch. Gradients are aggregated and redistributed. Machines are stalled until redistribution\\
        \textbf{Asynchronous} - Each machine computes minibatch. Gradients are aggregated and redistributed when possible. Machines continue with next minibatch before redistribution\\
        \textbf{Model parallel} - Model layers are distributed between machines (\textit{slower, than running on a single GPU})\\
        \textbf{Model parallel Pipeline} - \textbf{Model parallel} but when $2$nd GPU processes $1$st micro-batch $1$st GPU starts with the $2$nd one \dots Backwards pass starts from last GPU when all micro-batches are done.
    \end{mdframed}

    \section{Vision Models}
    \begin{mdframed}[style=eqbox]
        \textbf{Segmentation} - Extract patch, classify center pixel - Fully convolutional network can classify all pixels at once, up-sampling\\
        \textbf{Upconvolution} - Fractional Strided Convolution\\
        \textbf{U-Net} - Down-sampling + Max pooling for classification, Up-sampling + Skip connection to regain detail
        \subsection{Autoencoders}
        Project input into low dimensional (latent) space $\mathcal{Z}$ so that original data can be recovered.\\
        \textbf{VAE} - Variational Autoencoder\\
        $Z$ has a known distribution $p_z(z \mid x)$ (normal distribution), $p_z$ is forced to be close to $\mathcal{N}(0, \mat{I})$ so that distributions of different $x$ overlap.\\
        \textbf{Denoising Autoencoder} - High dimensional latent space would usually lead to identity $\implies$ parts of the input or latent representation are removed or noise is added to force robust representation.\\
    \end{mdframed}

    \begin{mdframed}[style=eqbox]
        \subsection{Generative Models}
        \textbf{GAN} - Generative Adversarial Network - \textit{PRO:} high-quality, fast - \textit{CON:} Implicit density, training unstable, only external measures for comparing\\
        \textbf{VAE} - \textit{PRO:} Explicit density, fast - \textit{CON:} lower bound on density, coverage / quality trade-off\\
        \textbf{Diffusion Models} - \textit{PRO:} flexible architecture, stable training, latent variable, high reselution - \textit{CON:} Expensive due to long markov chain
        \begin{align*}
            q(x_t \mid x_{t-1}) &= \mathcal{N}(\sqrt{1-\beta_t}x_{t-1},\beta_t \mat{I})
        \end{align*}
    \end{mdframed}
\end{multicols}
\end{document}
